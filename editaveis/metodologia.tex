\chapter[Metodologia]{Metodologia}

O interesse desse capítulo é responder como os objetivos desse trabalho serão
alcançados e como a solução do problema de pesquisa será abordada. As seguintes
seções são apresentadas: seção \ref{sec:classificacao}, apresenta o tipo de
metodologia de pesquisa utilizada no desenvolvimento desse trabalho; seção
\ref{sec:modelagem}, especifíca o planejamento, as atividades, e fluxo de
execução; seção \ref{sec:cronograma}, apresenta o cronograma; seção
\ref{sec:provaconceito}, detalha a prova de conceito.

\section{Classificação da Pesquisa}
\label{sec:classificacao}

Para a classificação de um tipo de pesquisa, é necessária a definição clara de
critérios baseados nos objetivos e procedimentos a serem seguidos
\cite{gil2002}. Os tipos de pesquisa baseados nos objetivos são usualmente
classificados em três grupos: exploratória, descritiva e explicativa
\cite{gil2002}.

A pesquisa exploratória é definida por Theodorson e Theodorson
\cite{theodorson1970} como sendo um estudo preliminar com o objetivo principal
de se familiarizar com o tópico abordado, de modo que o estudo posterior possa
ser realizado com maior compreensão e precisão. Theodorson e Theodorson
também esclarecem que a pesquisa exploratória permite que o pesquisador defina
o seu problema de pesquisa e formule hipóteses mais precisas, tornando possível
a escolha de técnicas mais adequadas e gerando alertas para quais as potenciais
dificuldades e sensibilidades da área.

De acordo com Fonseca \cite{fonseca2002}, a pesquisa descritiva é utilizada
quando o pesquisador tem conhecimento prévio do assunto e possui interesse em
descrever um fenômeno. A partir desse conhecimento prévio, o pesquisador pode
formular hipóteses, podendo confirmá-las ou não.

Já a pesquisa explicativa tem como objetivo central identificar fatores que
determinam ou contribuem para a ocorrência de fenômenos. Esse tipo de pesquisa
explica o porquê, a razão, o motivo de acontecimentos \cite{gil2002}.

Dado os objetivos de estudo e contexto desse trabalho, a metodologia de pesquisa escolhida foi a pesquisa exploratória.

\section{Modelagem do Fluxo de Atividades}
\label{sec:modelagem}

O Diagrama apresentado na Figura \ref{figura:diagram}, desenvolvido utilizando
o BPMN2 \textit{Modeler Plugin} para Eclipse \cite{eclipse}, ilustra a condução
das atividades requeridas para o desenvolvimento do TCC 1, e as planejadas para
o TCC 2. As descrições das atividades presentes no diagrama são apresentadas em
seguida.


\begin{figure}[H]
  \centering
  \includegraphics[width=16cm]{figuras/diagram2}
  \caption{Fluxo de atividades do TCC}
  \label{figura:diagram}
\end{figure}

Atividades mapeadas para o TCC 1:

\textbf{Elaborar referencial teórico:} teve como objetivo construir o
conhecimento base para a elaboração do trabalho. Para tanto, considerou-se a
literatura, com a investigação de livros e artigos de autores renomados da área.

\textbf{Escolher ferramenta MultiAgentes para \textit{Mobile Social Games}:}
teve como objetivo a pesquisa, e a escolha de uma ferramenta que aplicasse o
paradigma MultiAgentes no contexto de \textit{Mobile Social Games}. Essa
pesquisa obteve como resultado uma única ferramenta, o AMUSE, descrito
detalhadamente no capítulo \ref{chapter:suporteTecnologico} de Suporte
Tecnológico.

\textbf{Analisar características da ferramenta:} teve como objetivo analisar as
características e restrições da ferramenta escolhida, no caso, o AMUSE.

\textbf{Elaborar proposta:} teve como objetivo a definição de uma proposta que
contemplasse os objetivos deste trabalho de conclusão de curso. Essa proposta
foi composta de uma visão mais geral, orientada pelos objetivos de pesquisa; e
uma visão mais técnica, orientada por uma baseline arquitetural.

\textbf{Levantar suporte tecnológico:} teve como objetivo o levantamento das
das ferramentas de apoio necessárias para a elaboração desse trabalho.

\textbf{Implementar prova de conceito, seção \ref{sec:provaconceito}:} teve
como objetivo visando maior embasamento bem como possibilitando avaliar a
viabilidade técnica desse trabalho. Essa prova de conceito representa ainda
parte do desenvolvimento da proposta.

\textbf{Documentar e analisar resultados:} teve como objetivo a formalização dos
resultados obtidos no desenvolvimento da prova de conceito bem como alcançados
até o momento.

\textbf{Finalizar escrita do TCC 1:} teve como objetivo o refinamento do
trabalho, e adição de observações pertinentes para entrega à banca avaliadora.

Atividades mapeadas para o TCC 2:

\textbf{Realizar correções propostas pela banca:} possui como objetivo a
correção dos pontos levantados pela banca avaliadora.

\textbf{Implementar \textit{Mobile Social Game}:} possui como objetivo dar
continuidade à implementação da proposta.

\textbf{Documentar e analisar resultados:} possui como objetivo a documentação e
a análise dos resultados obtidos com o desenvolvimento da proposta, e evolução
do projeto.

\textbf{Finalizar escrita do TCC 2:} possui como objetivo o refinamento do
trabalho, e adição de observações pertinentes para entrega à banca avaliadora.

\textbf{Publicar jogo na Google Play:} possui como objetivos a divulgação e a
monetização da proposta desenvolvida através da publicação na plataforma Google
Play.


\section{Cronograma de Pesquisa}
\label{sec:cronograma}

O cronograma seguido para o desenvolvimento desse trabalho pode ser observado
nas Tabela 1 e Tabela 2. A Tabela 2, referente ao cronograma do TCC 2,
portanto, passível de mudanças de acordo com o andamento do trabalho.

%Cronograma 1
\begin{table}[H]
\centering
\resizebox{\textwidth}{!}{%
\begin{tabular}{|l|l|c|c|c|l|c|}
\hline
\multicolumn{1}{|c|}{\textbf{Atividade}} & \multicolumn{1}{c|}{\textbf{Janeiro}} & \textbf{Fevereiro} & \textbf{Março} & \textbf{Abril} & \textbf{Maio} & \textbf{Junho} \\ \hline
Levantamento Bibliográfico & \multicolumn{1}{c|}{X} & X & X & \multicolumn{1}{l|}{} &  &  \\ \hline
Definição do Escopo &  & X & X & \multicolumn{1}{l|}{} &  &  \\ \hline
Levantamento do Suporte Tecnológico &  & X & X & \multicolumn{1}{l|}{} &  &  \\ \hline
Implementar Prova de Conceito &  &  & X & X & \multicolumn{1}{c|}{X} &  \\ \hline
Análise de Resultados &  & \multicolumn{1}{l|}{} & X & \multicolumn{1}{c|}{X} & \multicolumn{1}{c|}{X} &  \\ \hline
Escrita do TCC 1 &  & \multicolumn{1}{l|}{} & X & X & \multicolumn{1}{c|}{X} & X \\ \hline
Apresentação do TCC 1 &  & \multicolumn{1}{l|}{} & \multicolumn{1}{l|}{} &  &  & \multicolumn{1}{c|}{X} \\ \hline
Realizar correções requisitadas pela banca &  & \multicolumn{1}{l|}{} & \multicolumn{1}{l|}{} & \multicolumn{1}{l|}{} &  & \multicolumn{1}{c|}{X} \\ \hline
\end{tabular}
}
\caption{Cronograma TCC 1}
\label{my-label}
\end{table}


% Cronograma 2
\begin{table}[H]
\centering
\resizebox{\textwidth}{!}{%
\begin{tabular}{|l|l|l|c|c|}
\hline
\multicolumn{1}{|c|}{\textbf{Atividade}}   & \multicolumn{1}{c|}{\textbf{Junho}} & \multicolumn{1}{c|}{\textbf{Julho}} & \textbf{Agosto}       & \textbf{Setembro} \\ \hline
Implementar Mobile Social Game             & \multicolumn{1}{c|}{X}              & \multicolumn{1}{c|}{X}              & X                     &                   \\ \hline
Documentar e Analisar Resultados           &                                     &                                     & X                     & X                 \\ \hline
Escrita do TCC 2                           &                                     &                                     & X                     & X                 \\ \hline
Apresentação do TCC 2                      &                                     &                                     & \multicolumn{1}{l|}{} & X                 \\ \hline
Realizar correções requisitadas pela banca &                                     &                                     & \multicolumn{1}{l|}{} & X                 \\ \hline
\end{tabular}
}
\caption{Cronograma TCC 2}
\label{my-label}
\end{table}

\section{Prova de Conceito}
\label{sec:provaconceito}

Dado o contexto de estudo desse projeto, é necessária a integração de várias
plataformas que utilizam tecnologias distintas. A avaliação dos possíveis
riscos de integração dessas plataformas é pertinente, no atual momento do
projeto, pois ocorre em um estágio inicial de desenvolvimento.

Para avaliar a viabilidade técnica do desenvolvimento da proposta, foi elaborada
uma prova de conceito, que possui como objetivo principal a integração das
várias plataformas presentes na proposta desse projeto. As subseções
\ref{subsec:descconceito} e \ref{subsec:configambiente} detalham a Descrição da
Prova de Conceito e a Configuração do Ambiente de Desenvolvimento,
respectivamente.

  \subsection{Descrição da Prova de Conceito}
  \label{subsec:descconceito}
  A pesquisa exploratória realizada, juntamente com o referencial teórico
  levantado, revelou que será necessária a utilização das plataformas: JADE, seção \ref{sec:jadePlatform}; WADE, seção \ref{sec:wadePlatform}; AMUSE, seção \ref{sec:amusePlatform}; e Android, seção \ref{sec:android} para a aplicação do paradigma MultiAgentes em \textit{Mobile Social Games}.

  \begin{figure}[h]
    \centering
    \includegraphics[width=5cm]{figuras/login_screen}
    \caption{Protótipo de tela de login para Prova de Conceito}
    \label{figura:login_screen}
  \end{figure}

  Para validar as integrações entre as plataformas mencionadas anteriormente,
  proporcionar a configuração do ambiente de desenvolvimento, e criar um
  subsídio reutilizável para o desenvolvimento da proposta, capítulo
  \ref{chapter:proposta}, a seguinte Prova de Conceito foi desenvolvida: uma
  versão inicial de controle de autenticação de usuários através de nome e
  senha, conforme Figura \ref{figura:login_screen}.

  \begin{figure}[h]
    \centering
    \includegraphics[width=10cm]{figuras/platforms_overview}
    \caption{Visão arquitetural alto nível da Prova de Conceito}
    \label{figura:platforms_overview}
  \end{figure}

  Uma visão arquitetural alto nível da Prova de Conceito pode ser vista na
  Figura \ref{figura:platforms_overview}. Essa Prova de Conceito possui a
  integração entre as plataformas, e utiliza o banco de dados SQL H2 Database
  \cite{h2db}, devido à sua simplicidade de integração. No entanto, esse banco
  pode ser alterado, caso necessário.

  O protótipo de aplicativo Android é disposto na Figura
  \ref{figura:platforms_overview}, e descrito através do diagrama de classes na
  Figura \ref{figura:class_diagram_prototype}. Ele utiliza as bibliotecas AMUSE
  \textit{client} e JADE Android para o controle, e comunicação do processo de
  autenticação do usuário com a plataforma local AMUSE.

  Esse processo de autenticação é realizado através das classes
  \textit{DefaultLoginManager}, \textit{CallbackManager} e
  \textit{AmuseClient}. A classe \textit{AmuseClient} é responsável por
  estabelecer a comunicação com a plataforma AMUSE. A classe
  \textit{CallbackManager} é utilizada para o tratamento das funções de
  \textit{callback}. Por fim, a classe \textit{DefaultLoginManager} é uma classe
  utilitária que fornece o método de autenticação simples através de nome e
  senha.

  Esta primeira versão de controle de autenticação permitiu que os fluxos de
  integração entre as plataformas fossem validados, e o ambiente de
  desenvolvimento configurado. Além de ser um módulo reutilizável para a
  proposta.

  O tópico a seguir descreve os passos necessários para a configuração do
  ambiente de desenvolvimento, utilizando como base as ferramentas de
  desenvolvimento, seção \ref{sec:tecferramentasdesenvolvimento}.

  \begin{figure}[h]
    \centering
    \includegraphics[width=16cm]{figuras/class_diagram_prototype}
    \caption{Diagrama de classe do protótipo }
    \label{figura:class_diagram_prototype}
  \end{figure}

  \subsection{Configuração do Ambiente de Desenvolvimento}
  \label{subsec:configambiente}

  Para realizar o desenvolvimento da Prova de Conceito, utilizou-se de uma
  máquina com as seguintes especificações: processador 2.6 GHz Intel Core i5,
  memória ram de 16 GB 1600 MHz DDR3 e sistema operacional MacOS Sierra 10.12.5.

  O primeiro passo para a configuração do ambiente foi a verificação da
  existência do Java \cite{java}, e Apache Ant \cite{ant} na máquina de
  desenvolvimento. Visto que as plataformas utilizadas nesse projeto
  necessitam dessas duas tecnologias para a configuração base.

  Após a verificação da existência do Java, seguiu-se os passos disponíveis para
  a configuração do AMUSE. Tais passos estão disponíveis no tutorial de
  configuração, disponível em:
  \url{http://jade.tilab.com/amuse/doc/Amuse-Startup-Guide.doc}.

  % Mostrar banco inicializado
  % Mostrar setup do wade e amuse

  % \subsection{Integração Entre Plataformas}
  % \label{subsec:integracaoplataformas}

\section{Considerações Finais do Capítulo}

O objetivo desse capítulo foi apresentar a metodologia utilizada no
desenvolvimento desse trabalho de conclusão de curso. Esse capítulo possui a
classificação do tipo de pesquisa, modelagem do fluxo de atividades,
cronograma, e prova de conceito para evolução da proposta desse TCC.
