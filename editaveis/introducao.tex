\chapter[Introdução]{Introdução}

Jogos \textit{multiplayer} atraem pessoas de diversos grupos sociais e faixas etárias. A possibilidade de interação com outros jogadores
é um dos maiores atrativos nesse tipo de jogo. Com o avanço da tecnologia, os tipos de dispositivos que comportam tais interações
estão cada vez mais acessíveis. Já é possível imergir em um ambiente virtual através de aplicativos para \textit{smartphones}. Com a
redução dos preços desses aparelhos, diversos usuários e desenvolvedores foram atraídos para a área de \textit{Mobile Social Games}.

\section{Contextualização}

Todos os dias, milhões de pessoas interagem entre si em ambientes virtuais conhecidos como
\textit{Massively- Multiplayer Online Role-Playing Games} (MMORPGs)\cite{nick2006}. As primeiras versões de MMORPGs eram baseadas em
interações através de comandos de texto. Esse primeiro estilo de MMORPG era conhecido como \textit{Multi-User Dungeons} (MUDs)\cite{edward1997}.
O menor custo de produção de dispositivos com maior capacidade de processamento, e conexões de internet mais velozes,
possibilitou que interfaces gráficas fossem adicionadas a tais jogos. No entanto, uma característica se manteve durante a evolução desses jogos,
as interações sociais entre os jogadores.

As relações sociais oriundas de tais jogos se tornaram mais fáceis através da popularização de redes sociais, como Facebook e GooglePlus.
Inúmeros \textit{Social Games} se utilizam das conexões presentes nessas redes para proporcionar uma maneira mais fácil de interação entre
os usuários. Jogos como Clash of Clans\cite{clashOfClans} e Candy Crush Saga\cite{candyCrush} se utilizam de tais redes sociais para facilitar
o encontro de jogadores.

Ambos os jogos mencionados acima são classificados como \textit{Social Games}. \textit{Social game} é um jogo \textit{online} que utiliza servi\c{c}os de redes sociais\cite{park2012}.
Esses serviços são utilizados para obter dados referentes aos diferentes tipos de conexões entre os usuários. Tais redes sociais podem ser de terceiros,
como as citadas anteriormente, ou podem ser gerenciadas pela própria empresa. A partir dessas observações, é possivel notar que
jogos \textit{online} são cada vez mais interações sociais do que experi\^encias individuais\cite{king2003}.

\textit{Social games} podem ser divididos em dois grandes grupos, \textit{mobile} e \textit{desktop}. Existem diferen\c{c}as entre \textit{Social Games} de dispositivos \textit{mobile} e \textit{Social Games} de computadores pessoais.
Ao contrário dos \textit{Social Games} de computadores pessoais, \textit{Mobile Social Games} são restringidos por limita\c{c}\~oes de \textit{hardware} e plataformas de \textit{software}, questões
referentes à usabilidade, entre outros\cite{yamakani2011}.

Através de observações das restrições existentes em \textit{Mobile Social Games}, foi levantada a possibilidade de aplicação do paradigma MultiAgentes
no seu desenvolvimento. Os sistemas MultiAgentes são sistemas
compostos por múltiplos elementos computacionais que interagem entre si, conhecidos como agentes \cite{wooldridge2009}.
Uma das vantagens na utilização de agentes é que eles provêem características que proporcionam comunicação dinâmica em
grupos grandes de usuários\cite{bergenti2015}. Agentes também são componentes de software que podem ser reutilizáveis, reduzindo o tempo de
implementação\cite{bergentiHuhns2014}.

De acordo com o levantamento bibliográfico realizado, foi possível a identificação de uma única proposta de aplicação do paradigma MultiAgentes no contexto de \textit{Mobile Social Games}. Tal proposta refere-se ao \textit{framework} AMUSE (\textit{Agent-based Multi-User Social Environment})
\cite{amuse}. O AMUSE é uma plataforma de código aberto, ainda em desenvolvimento, para \textit{Social Games}.
Ela possui o formato PaaS (\textit{Platform as a Service})\cite{bergenti2015}, uma ferramenta que possibilita provedores de serviço,
como jogos e comunidades de portais, a reduzir da carga de implementação de funcionalidades
comuns a vários jogos\cite{bergenti2015}.

\section{Justificativa}

A decisão de trabalhar uma abordagem MultiAgentes em \textit{Mobile Social Games} foi instigada devido às características desse paradigma,
dentre as quais destaca-se o fato dos Sistemas MultiAgentes proporcionarem melhorias na dinâmica da coordenação de grandes grupos
de usuários, permitindo a redução de tempo no desenvolvimento da solução. Tal fato é evidenciado pelos autores Fabio Bellifemine, Giovanni Caire e Dominic Greenwood
\cite{fabio2007}. Adicionalmente, tem-se que a redução de tempo de desenvolvimento corresponde à uma característica desejada
para produção de software\cite{bergenti2015}.


\section{Questão de Pesquisa}

Esse TCC procura responder a seguinte questão de pesquisa: Como lidar com \textit{Mobile Social Games} utilizando uma abordagem MultiAgentes?


\section{Objetivos}

Esta seção define os objetivos geral e específicos referentes ao trabalho.

    \subsection{Objetivo Geral}

        Investigar uma abordagem MultiAgentes para \textit{Social Game} explorando o desenvolvimento de um \textit{Social Game} assíncrono para plataforma \textit{mobile} Android.

    \subsection{Objetivos Específicos}

        Os objetivos específicos abordados por esse TCC são:

        \begin{itemize}
          \item Realizar uma revisão bibliográfica visando identificar problemas comumente encontrados no desenvolvimento de \textit{Mobile Social Games} (assíncronos);
          \item Acordar soluções candidatas - orientadas a Sistemas MultiAgentes - para lidar com os problemas identificados no objetivo anterior;
          \item Realizar provas de conceito, as quais serão  baseadas em ciclos evolutivos de desenvolvimento, seguidos de pesquisa-ação;
          \item Compilar os resultados obtidos ao longo do processo investigativo usando uma abordagem híbrida quantitativa e qualitativa;
          \item Aprofundar os conhecimentos em Engenharia de Software, motivado pelo domínio de \textit{Mobile Social Games}, orientando-se por padrões arquiteturais, metodologias, testes e outras boas práticas.
        \end{itemize}

\section{Organização dos Capítulos}

Esse TCC está organizado em seis capítulos, sendo este primeiro a Introdução. Os demais capítulos são resumidos brevemente a seguir:

  \begin{itemize}
    \item \textbf{Referencial Teórico:} este capítulo explana sobre conceitos importantes de \textit{Mobile Social Games}, Engenharia de Software e Sistemas MultiAgentes;
    \item \textbf{Suporte Tecnológico:} explica sobre as tecnologias que serão utilizadas nesse trabalho para suporte e desenvolvimento da prova de conceito. Tais tecnologias são \textit{frameworks}, sistemas operacionais, ferramentas de controle de versão entre outras;
    \item \textbf{Metodologia:} descreve a modalidade de pesquisa que será utilizada, o planejamento e descrição das atividades que serão realizadas, o e cronograma do trabalho;
    \item \textbf{Proposta:} descreve de forma detalhada a proposta de pesquisa e desenvolvimento desse trabalho;
    \item \textbf{Resultados Obtidos:} apresenta os resultados obtidos até o momento, e as futuras atividades a serem executadas.
  \end{itemize}
