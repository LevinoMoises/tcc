\chapter[Proposta]{Proposta}

A proposta desse trabalho consiste em desenvolver um \textit{Mobile Social Game} assíncrono utilizando uma abordagem MultiAgentes para a plataforma Android. A partir do referencial teórico levantado, e ferramentas MultiAgentes disponíveis para o contexto de \textit{Mobile Social Games}, foi definido a utilização do AMUSE como plataforma base de desenvolvimento.

O \textit{Mobile Social Game} deve ser do estilo \textit{Online Trading Card Games}; Implementar funcionalidades de: organização de partidas, síncronização de jogadas, controle de conexão \textit{peer to peer}; Seguir o modelo de monetização \textit{Freemium} e; Possuir características que o definam como \textit{Social Game}, sendo elas: utilizar uma plataforma de serviços, ou rede social, possuir atividades de interação social como competição e colaboração, possuir interações síncronas e assíncronas, e outras interações entre usuários que ajudem na adoção e manutenção do jogo.


%% Definir o Estilo do jogo


\section{Arquitetura do Sistema}

%% Arquitetura igual ao do AMUSE


\section{Desenvolvimento do Aplicativo}


%CODE MODAFUCKEAR

\section{Considerações Finais do Capítulo}
