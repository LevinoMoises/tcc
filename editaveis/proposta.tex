\chapter[Proposta]{Proposta}
\label{chapter:proposta}

A pesquisa exploratória realizada, e o referencial teórico levantado,
expuseram  a plataforma AMUSE como única ferramenta de desenvolvimento de
\textit{Mobile Social Games} que utiliza uma abordagem MultiAgentes. Levando em
consideração essa condição, características desejadas fornecidas pela
plataforma, e objetivo desse trabalho, ela será a ferramenta base para o
desenvolvimento.

Até o momento da escrita desse projeto, os agentes da plataforma AMUSE que
aplicam os padrões de \textit{design} de jogos não estão completos.
Por exemplo, os agentes GRA(\textit{Games Room Agent}) e MMA(\textit{Match
Tracer Agent}) só permitem que um tipo de interação entre jogadores seja
realizada, sendo ela assíncrona, ou síncrona \cite{bergenti2013}.

Levando em consideração as restrições presentes na ferramenta AMUSE, e
consequentemente a abordagem MultiAgentes em \textit{Mobile Social Games}, a
proposta desse trabalho é restringinda a um conjunto de características e
padrões de \textit{design} predefinidos, e apresentados na próxima seção.

\section{Definição da Proposta}

A proposta desse trabalho consiste no desenvolvimento de um \textit{Mobile
Social Game} no estilo \textit{collectible card game} para a plataforma
Android.

Nessa proposta, os jogadores serão dispostos em uma arena em que
poderão desafiar outros jogadores para um duelo. Esse duelo consiste em uma
batalha de cartas, em turno, entre dois jogadores. Cada jogador utilizará um
conjunto de cartas próprias, e sairá vencedor caso consiga reduzir os
pontos do seu oponente a zero.

Ao final de cada duelo, o jogador vencedor receberá pontos em um sistema de
\textit{ranking} global, e terá a opção de trocar uma carta própria, com uma
carta do oponente.

Além desse sistema de duelos, a proposta abrange um conjunto de características
relacionadas a \textit{Mobile Social Games}, padrões de \textit{design} de
jogos, e modelo de monetização.

As características que definem um jogo como \textit{Mobile Social Games}
são variadas, e descritas na seção \ref{sec:refmobilesocialgames}. Nesse
trabalho, as seguintes características foram selecionadas para aplicação na
proposta:

\begin{itemize}
  \item Interações assíncronas entre os jogadores;
  \item Formação de laços de amizade assimétrica;
  \item Laço fraco na formação de relações sociais;
\end{itemize}

No contexto de \textit{design} de jogos, os seguintes padrões foram
selecionados:

\begin{itemize}
  \item \textit{Ticket-based game};
  \item \textit{Collectible card game};
  \item \textit{Public player statistics};
\end{itemize}

Para a monetização dessa proposta, o modelo \textit{Freemium} será aplicado.
Esse modelo permite que novos usuários possam jogar sem a necessidade de
dispêndio inicial.

A próxima seção apresenta uma visão arquitetural da disposição dos elementos
que compõe essa proposta, e ilustra os pontos de aplicação do paradigma
MultiAgentes.

\section{Arquitetura do Sistema}
%% Arquitetura igual ao do AMUSE
TODO

\section{Condução do Desenvolvimento do Projeto}
TODO

\section{Considerações Finais do Capítulo}

TODO
