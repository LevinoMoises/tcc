\chapter[Proposta]{Proposta}

A pesquisa exploratória realizada, e o referencial teórico levantado, expuseram a
plataforma AMUSE como única ferramenta de desenvolvimento de \textit{Mobile Social Games}
que utiliza uma abordagem MultiAgentes. Levando em consideração essa condição,
características desejadas fornecidas pela plataforma, e objetivo desse trabalho, ela
será a ferramenta base para o desenvolvimento.

Até o momento da escrita desse projeto, o estado de implementação dos agentes bases da
plataforma AMUSE que aplicam os padrões de \textit{design} de jogos não está completo.
Por exemplo, os agentes GRA(\textit{Games Room Agent}) e MMA(\textit{Match Tracer Agent})
só permitem que um tipo de interação entre jogadores seja realizada,
sendo ela assíncrona, ou síncrona \cite{bergenti2013}.

Levando em consideração as restrições presentes na ferramenta AMUSE, e consequentemente a
abordagem MultiAgentes em \textit{Mobile Social Games}, a proposta desse trabalho
é restringinda a um conjunto de características e padrões de \textit{design} predefinidos.

\section{Definição da Proposta}

A proposta desse trabalho consiste em desenvolver um \textit{Mobile Social Game} para a
plataforma Android com as seguintes características base de \textit{Mobile Social Games}:

\begin{itemize}
  \item Interações assíncronas entre os jogadores;
  \item Formação de laços de amizade assimétrica;
  \item Laço fraco na formação de relações sociais;
\end{itemize}

Os padrões de \textit{design} de jogos que serão aplicados a princípio são:

\begin{itemize}
  \item \textit{Ticket-based game};
  \item \textit{Public player statistics};
\end{itemize}

O estilo do jogo(\textit{gameplay}) será baseado no \textit{Online Trading Card Games}.
E o modelo de monetização a ser aplicado é o \textit{Freemium}.


\section{Arquitetura do Sistema}

TODO
%% Arquitetura igual ao do AMUSE


\section{Desenvolvimento do Aplicativo}

TODO

%CODE MODAFUCKEAR

\section{Considerações Finais do Capítulo}

TODO
