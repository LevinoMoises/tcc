\chapter[Referencial Teórico]{Referencial Teórico}

Este capítulo apresenta conceitos pertinentes aos tópicos de maior relevância
abordados nesse TCC. Os tópicos abordados são apresentados através das seções:
seção \ref{sec:refmobilesocialgames}, acorda a base teórica
de \textit{Mobile Social Games}; seção \ref{sec:refengdesoftware},
descreve o referencial teórico de Engenharia de Software; seção
\ref{sec:refsistemasmultiagentes}, apresenta a base teórica de
Sistemas MultiAgentes.

\section{Mobile Social Games}
\label{sec:refmobilesocialgames}

A área de \textit{Mobile Social Games} possui diversos jogos com os mais
variados estilos. Esses estilos podem ser subdivididos em dois grandes grupos:
\textit{Hard-Core Games} e \textit{Casual Games}. Ambos possuem
características em comum que os definem como \textit{Social Games}, mas também
possuem peculiaridades referentes ao tipo de interação social presente no
jogo. Os \textit{Hard-Core Games} são caracterizados por serem ambientes de
extrema interação social, onde diversas oportunidades para criação de laços de
amizade são presentes \cite{cole2007}. \textit{Casual Games}, por outro lado,
apresentam interações mais casuais em que laços de amizade são mais fracos
\cite{ricchetti2015}.

    \subsection{Definição de Mobile Social Games}
Existem diversas definições para \textit{Social Game}. Entretanto, todas
possuem características semelhantes. De acordo com os autores Bergenti, Caire,
Gotta, Park e Lee \cite{bergenti2013} \cite{park2012}, \textit{Social Game} é
todo o jogo que utiliza uma plataforma, ou serviços, de uma rede social.
Fields \cite{fields2014} define que \textit{Social Game} é um jogo em que as
interações entre usuários ajudam na adoção e manutenção de jogadores
conectados através de uma rede social externa, facilitando o alcance desses
objetivos. Já Matt Ricchetti \cite{ricchetti2015} faz uma definição mais
social. Ele define \textit{Social Game} como sendo uma comunidade que possui
várias interações presentes em relações sociais, tais como: competição,
colaboração, rebelião, inveja e compaixão.

Matt Ricchetti \cite{ricchetti2015} vai além, e também define três características para divisão de \textit{Social Games}. Essas características são baseadas no tipo de interação social e sincronia do jogo. As três características relativas a \textit{Social Games} são:

\begin{itemize}
  \item Interação síncrona vs assíncrona entre usuários. As interações ocorrem simultaneamente em tempo real, ou em tempos diferentes, em um estilo de turnos?;
  \item Formação simétrica vs assimétrica de relações sociais. Formar uma relação social requer entrada de dados de ambos os participantes, ou pode ser formada de maneira unilateral?;
  \item Laço forte vs fraco na formação de relações sociais. As relações sociais tendem a ser fortes e duradouras, ou tendem a ser fracas e transitórias?
\end{itemize}

As três definições apresentadas são complementares e utilizadas em conjunto
durante esse trabalho. As características alvo desse trabalho são descritas no
capítulo \ref{chapter:proposta}.
Por sua vez, \textit{Mobile Social Games} refere-se a \textit{Social Games} que são desenvolvidos para plataformas \textit{mobile}, como Android, iOS e Windows Phone.

  \subsection{Perspectiva Histórica de Mobile Social Games}

Os primeiros \textit{Social Games}, assim como os demais jogos de computador,
eram bastante simples. Suas interfaces gráficas eram simuladas por texto e a
quantidade de opções disponíveis eram limitadas. As primeiras versões de
\textit{Social Games} são conhecidas como \textit{Multi-User Dungeons}
(MUDs)\cite{edward1997}. Os MUDs possuíam seu estilo de jogo baseado em turnos,
mas também permitiam que usuários interagissem de maneira síncrona ou
assíncrona, em ambientes com diversas temáticas, desde guerra no espaço, até o
velho oeste americano. Os MUDs eram executados em redes sociais conhecidas como
\textit{Bulletin Board Systems} (BBS). Tais redes sociais são baseadas em
sistemas de texto que permitiam a interação entre usuários através de uma rede
de telefone \cite{fields2014}. \textit{Trade Wars}, Figura
\ref{figura:tradeWars}, é um exemplo de MUD executado em BBS. O \textit{Trade
Wars} foi lançado em 1984 e possuia a temática de guerra no espaço. Nesse jogo,
os usuários podiam competir e formar alianças para conquistar recursos.

\begin{figure}[h]
  \centering
  \includegraphics[width=10cm]{figuras/trade_wars.png}
  \caption{MUD \textit{Trade Wars} \cite{fields2014}}
  \label{figura:tradeWars}
\end{figure}

A redução no custo de produção de dispositivos com maior capacidade de
processamento, e conexões de internet mais velozes e estáveis, permitiu que
interfaces gráficas em nível 2D e 3D, e o acesso aos \textit{Social Games}
fossem expandidos em diversas plataformas, como \textit{PC} e \textit{Mobile}.

A evolução dos MUDs conduziu ao desenvolvimento dos \textit{Massively-
Multiplayer Online Role-Playing Games} (MMORPGs). Os MMORPGs possuem
características que os definem como \textit{Social Games}. Nesse estilo de
jogo, é possível a troca de mensagens entre usuários através de uma rede
interna, e muitas vezes externa, \textit{chat} em tempo real, formação de
grupos para cooperação e vários outros estilos de organizações sociais
\cite{fields2014}.

Um dos MMORPG de maior sucesso para a plataforma PC é o \textit{World of
Warcraft}, Figura \ref{figura:wow}. O \textit{World of Warcraft} foi
desenvolvido pela \textit{Blizzard Entertainment}$^{\tiny{©}}$
em 2004, e possui, até o momento da escrita desse trabalho, o título de MMORPG
mais rentável da história \cite{thurau2010} \cite{omer2015}. \textit{World of
Warcraft} é baseado em um mundo medieval de fantasias, no qual os jogadores
controlam personagens que podem evoluir de nível, desenvolver habilidades,
participar de guerras, entre outras atividades. As atividades sociais do
\textit{World of Warcraft} são constantes no jogo. Jogadores interagem em
ações de dança, festas, grupos de \textit{chat} e formação de grupos para
tarefas que não são possíveis de maneira individual, como conquista de itens
especiais \cite{thurau2010} \cite{nardi2006}.

\begin{figure}[h]
  \centering
  \includegraphics[width=10cm]{figuras/wow.jpg}
  \caption{MMORPG \textit{World of Warcraft} \cite{kotaku}}
  \label{figura:wow}
\end{figure}

Os MMORPG começaram na plataforma \textit{mobile} em 2003 com o lançamento do
TibiaME, versão \textit{mobile} do MMORPG Tibia para PC \cite{tibiaME}. O
TibiaME é um MMORPG no estilo de fantasia medival em que os jogadores podem
evoluir o nível dos seus personagens, participar de guerras e grupos sociais.
As versões iniciais do TibiaME eram destinadas aos primeiros dispositivos
móveis com suporte para J2ME (\textit{Java 2 Platform, Micro Edition})
\cite{tibiaMEhistory}. As primeiras versões possuiam gráficos em 2D, Figura
\ref{figura:tibiaME}, devido às restrições de hardware presentes nos primeiros
\textit{smartphones}, e necessitavam de acesso constante à internet, sendo
essa realizada inicialmente através das primeiras redes 2G.

A evolução do hardware em \textit{smartphones}, novas plataformas, redes
móveis mais estáveis e velozes, e paralelamente a criação de novas redes
sociais, como o \textit{Facebook}, possibilitaram o desenvolvimento e a
melhoria de vários novos tipos de \textit{Mobile Social Games}, além dos
MMORPGs \cite{fields2014}. Alguns desses estilos de \textit{Mobile Social
Games} são: \textit{Turn-Based Building Games}, \textit{Simulation Games},
\textit{Virtual Worlds}, \textit{Non-Persistent Action and RTS Games} e
\textit{Online Trading Card Games} \cite{fields2014}. Atualmente, vários
\textit{Mobile Social Games} estão disponíveis em lojas de aplicativos de
todas as grandes plataformas, entre elas: Google Play$^{\tiny{®}}$, Apple
Store$^{\tiny{®}}$ e Windows Phone Store$^{\tiny{®}}$.

\begin{figure}[h]
  \centering
  \includegraphics[width=5cm]{figuras/tibiaME}
  \caption{TibiaME$^{\tiny{®}}$. Primeiro MMORPG para plataforma \textit{Mobile} \cite{tibiaMEhistory}}
  \label{figura:tibiaME}
\end{figure}

  \subsection{Problemas Ligados a Mobile Social Games}

Os \textit{smartphones} são dispositivos que estão presentes no cotidiano das
pessoas. Os mesmos são produzidos por diferentes empresas, possuem sistemas
operacionais distintos, tamanhos de telas variados, dentre outras
especificações de hardware e software. Essa essência portátil, fragmentação de
sistemas operacionais e hardware, rede móvel precária e capacidade de
processamento menor que computadores pessoais, são raízes de vários problemas
encontrados em \textit{Mobile Social Games}.

De acordo com o \textit{International Data Corporation} \cite{idc}, o
\textit{marketshare} de sistemas operacionais para \textit{smartphones} é
amplamente ocupado pelo Android. No segundo quadrimestre de 2015, o sistema
operacional Android possuia 82.8\% do mercado de \textit{smartphones}, enquanto
que o segundo maior sistema operacional, Apple iOS, possuia apenas 13.9\% do
mercado \cite{idc}, conforme Figura \ref{figura:market_share}. Esses dados
poderiam indicar uma maior facilidade para o desenvolvimento de aplicativos
para \textit{smartphones}, dado que a maior parte dos dispositivos possui o
sistema operacional Android. No entanto, essa análise é incorreta. Mesmo
possuindo um \textit{marketshare} muito inferior ao Android, o iOS ainda gera
uma receita maior que o seu concorrente \cite{appAnnie}. Essa característica
faz com que desenvolvedores foquem em ambas as plataformas, visto a
oportunidade de geração de receita. Outro problema relacionado ao Android é a
sua própria fragmentação, vide Figura \ref{figura:android_chart}. Dados da
própria Google demonstram a existência de várias versões do Android, tamanhos
de telas e suporte a diferentes versões de \textit{engines} de vídeo
\cite{dashboardGoogle}. Todas essas diferenças aumentam os custos e dificultam
a criação de \textit{Mobile Social Games} que sejam compatíveis com a maior
parte do mercado.

\begin{figure}[h]
  \centering
  \includegraphics[width=12cm]{figuras/market_share}
  \caption{Quota de mercado de \textit{Smartphones} entre 2012 e 2015 \cite{idc}.}
  \label{figura:market_share}
\end{figure}

\begin{figure}[h]
  \centering
  \includegraphics[width=10cm]{figuras/android_chart}
  \caption{Fragmentação das versões do Android em 2015 \cite{dashboardGoogle}.}
  \label{figura:android_chart}
\end{figure}

  \subsection{Mercado de Mobile Social Games}

\textit{Mobile Games} tornaram-se uma das mais importantes plataformas para
jogadores e desenvolvedores de jogos. O mercado de \textit{Mobile Games}
demonstra crescimento contínuo e números expressivos. Na Figura
\ref{figura:gmgcMarket}, é possível analisar que em 2014, o mercado de
\textit{Mobile Games} apresentou um aumento de receita de 33\% e 55\%, em
\textit{smartphones} e \textit{tablets}, respectivamente. Em 2017, a
perspectiva é que \textit{Mobile Games} serão responsáveis por 38\% do mercado
global de jogos e receita prevista de \$40.4 bilhões de dólares \cite{gmgc}.

\begin{figure}[h]
  \centering
  \includegraphics[width=10cm]{figuras/gmgcMarket}
  \caption{Perspectiva de receita para o mercado \textit{Mobile} \cite{gmgc}}
  \label{figura:gmgcMarket}
\end{figure}

O mercado de \textit{Mobile Games} está em constante crescimento e possui
características que estão sendo analisadas para o mapeamento dos motivos para
o seu sucesso. Fields \cite{fields2014} revela algumas características
interessantes do mercado de \textit{Mobile Gaming} nos Estados Unidos da
América. Essas características são referentes à distribuição demográfica no
ano de 2013 \cite{fields2014}:

\begin{itemize}
  \item As mulheres representam 53\% dos usuários do mercado de \textit{Mobile Games};
  \item 50\% do jogadores possuem entre 25 e 30 anos;
  \item 57\% dos jogadores jogam diariamente, e 54\% jogam por mais de uma hora por dia;
  \item 30\% dos jogadores jogam na cama, e 16\% no ônibus ou metrô.
\end{itemize}

A receita gerada pelo mercado de \textit{Mobile Games} possui números
atraentes para os desenvolvedores, no entanto, a monetização de jogos não é
uma tarefa fácil. Atrair usuários para o seu jogo em um mercado repleto de
opções é uma tarefa difícil, especialmente quando se trata de uma empresa que
está iniciando, ou de porte pequeno. Dadas essas dificuldades, várias
estratégias de monetização de \textit{Mobile Games} são utilizadas
\cite{fields2014}:

\begin{itemize}
  \item \textit{Premium download};
  \item \textit{Subscriptions};
  \item \textit{Freemium}.
\end{itemize}

O modelo \textit{Premium download} caracteriza-se pela venda completa do jogo
para o usuário. O usuário compra o jogo em uma \textit{app store}, como
\textit{Google Play}$^{\tiny{®}}$ e \textit{Amazon Appstore}$^{\tiny{®}}$, e
obtém o jogo em sua completude, sem a necessidade de dispêndios futuros. Esse
modelo é normalmente utilizado por estúdios e franquias que já possuem uma
grande base de usuários, ou verba suficiente para atrair usuários através de
ações publicitárias.

O modelo \textit{Subscriptions} é baseado em assinaturas. O usuário recebe de
maneira gratuita os primeiros meses do jogo, sendo necessária uma assinatura
posterior para a continuação do jogo. Esse modelo é normalmente utilizado em
jogos que necessitam de um grande processamento em \textit{back-end}, como os
MMORPG. Esse modelo está em declínio devido ao crescimento de jogos de alta
qualidade que utilizam o próximo modelo, o \textit{Freemium}.

O modelo \textit{Freemium} pode ser subdividido em várias categorias, todas com
a mesma base: o jogo é gratuito. A monetização com esse modelo é feita através
da venda de itens, serviços e propagandas dentro do jogo. Os usuários possuem a
possibilidade de comprar itens para aumentar a velocidade de evolução no jogo,
liberar áreas indisponíveis e retirar propagandas indesejadas através de
pagamento.

Os modelos apresentados podem ser utilizados de maneira singular ou híbrida, o
que é necessário, é a análise de qual, ou quais, modelos se adequam ao estilo
de jogo implementado, tipo de público alvo, verba disponível para propaganda do
jogo, entre outros fatores.

Através dos dados apresentados, fica evidente que o mercado de \textit{Mobile
Games} está em um momento de crescimento e geração de oportunidades para
desenvolvedores independentes e empresas que já estão solidificadas no mercado
de jogos. Novas técnicas, princípios, metodologias e paradigmas podem ser a
solução para a melhoria do processo de desenvolvimento de \textit{Mobile
Social Games}. A próxima seção aborda tópicos ligados à Engenharia de
Software, como Sistemas MultiAgentes, que podem ser utilizados para abordar
alguns dos problemas relatados.

\section{Engenharia de Software}
\label{sec:refengdesoftware}

Dado os objetivos desse trabalho, faz-se necessário o embasamento teórico
quanto aos aspectos associados à Engenharia de Software. Dentre elas está a
utilização do paradigma MultiAgentes, testes unitários e automatizados para
Sistemas MultiAgentes e para a plataforma \textit{mobile} escolhida. Além de
técnicas para análise de qualidade de código fonte. Esse embasamento teórico
gera boas práticas como resultado. Tais boas práticas propiciam a aplicação de
melhorias no contexto de desenvolvimento e Engenharia de Software.

  \subsection{Sistemas MultiAgentes}
  \label{sec:refsistemasmultiagentes}

A visão baseada em agentes oferece suporte de ferramentas, técnicas, e
metáforas visando melhorar a maneira como as pessoas conceitualizam e
implementam diferentes produtos de software. Agentes são usados em uma
variedade crescente de aplicações — desde sistemas pequenos, como filtros
personalizados para emails, até grandes, complexos e críticos sistemas de
controle aéreo \cite{jade}. À primeira vista, pode parecer que tais sistemas
possuem pouco em comum. No entanto, esse não é o caso: em ambos, a abstração
chave utilizada é a de agentes \cite{jennings1998}.

De acordo com Wooldridge, Sistemas MultiAgentes (SMA) são sistemas compostos
por múltiplos elementos computacionais que interagem entre si, conhecidos como
Agentes. Agentes são sistemas computacionais com duas importantes capacidades.
Eles possuem, pelo menos em certo nível, capacidade de agirem de maneira
autônoma, de decidirem por si mesmos o que eles precisam fazer para satisfazer
os seus objetivos. Eles são capazes de interagir com outros agentes, não
simplesmente compartilhando dados,  mas se envolvendo em atividades análogas às
atividades sociais em que todos nós nos envolvemos diariamente: cooperação,
coordenação, negociação, e similares \cite{wooldridge2009}. As características
presentes em um Sistema MultiAgente são  \cite{jennings1998}:

\begin{itemize}
  \item Cada agente possui uma informação incompleta, ou capacidade para resolver um problema, ou seja, cada agente tem uma visão limitada;
  \item Não existe um sistema global de controle;
  \item Os dados são descentralizados e;
  \item A computação é assíncrona.
\end{itemize}

Alguns dos motivos pelo interesse em SMA são: habilidade de prover robustez e
eficiência; habilidade para permitir inter-operação entre sistemas legados; e
habilidade para resolver problemas em que \textit{data}, \textit{expertise} ou
controle são distribuídos \cite{jennings1998}.

As características e habilidades listadas anteriormente podem ser atingidas
através de arquiteturas distintas de Sistemas MultiAgentes.
Essas arquiteturas são descritas e abordadas no próximo tópico.

    \subsubsection{Tipos de Arquiteturas Orientadas a Agentes}

As arquiteturas de agentes são mecanismos fundamentais, subjacentes aos
componentes autônomos, que permitem comportamento efetivo e dinâmico em um
mundo real, e em ambientes abertos. As arquiteturas de agentes podem ser
divididas em quatro grupos principais: baseados em lógica, reativos,
crença-desejo-intenção(\textit{belief, desire, intention}), e em camadas (ou
híbrida) \cite{fabio2007}.

A arquitetura \textbf{baseada em lógica} possui sua fundação em técnicas
conhecidas de sistemas tradicionais em que o ambiente é representado de forma
simbólica e manipulado usando mecanismos de raciocínio \cite{fabio2007}.

Arquiteturas \textbf{reativas} implementam a tomada de decisão como um
mapeamento direto da situação para ação, baseando-se em estímulos
desencadeados por sensores de dados. Ao contrário da arquitetura
\textbf{baseada em lógica}, a arquitetura reativa não possue qualquer modelo
simbólico central, e portanto, não utiliza qualquer raciocínio lógico
simbólico complexo \cite{fabio2007}.

A arquitetura \textbf{crença-desejo-intenção}, ou BDI (\textit{belief, desire,
intention}), é baseada em três estruturas: crença (\textit{belief}), que é o
conhecimento que o agente possui do ambiente; desejo (\textit{desire}), que
representa os objetivos que ele deve alcançar; e intenção,  que representa as
tarefas a serem realizadas pelo agente, ou seja, planos estratégicos visando
alcançar os objetivos desejados (\textit{intention}) \cite{fabio2007}.

Por fim, a arquitetura em \textbf{camadas}, ou híbrida, permite tanto o
comportamento reativo quanto o deliberativo. Para permitir essa flexibilidade,
subsistemas são dispostos como camadas (horizontais e verticais) de maneira
hierárquica para acomodar ambos os tipos de agentes \cite{fabio2007}. A próxima
seção aborda aspectos relacionados à Testes de Software, como tipos de testes,
e frameworks de teste aplicados ao contexto desse trabalho.

  \subsection{Testes de Software}
  \label{subsec:testedesoftware}

Um dos objetivos desse trabalho é aprofundar os conhecimentos em Engenharia de
Software, orientando-se por padrões arquiteturais, testes e técnicas para
análise de qualidade de código fonte. Para isso, é necessária a apresentação de
referências para o desenvolvimento da base teórica. Essas referências
são apresentadas através das subseções: subseção \ref{subsec:testedesoftware},
apresenta as bases teóricas para testes de software, com foco em testes
unitários e de integração; subseção \ref{subsec:qualidadedesoftware} apresenta
as bases teóricas de qualidade de software, com foco em métricas de qualidade
de código fonte.

Teste de Software é um processo, ou uma série de processos, destinado(s) a
certificar-se que o código de computador faça o que foi projetado para fazer,
e inversamente, não faça algo não planejado \cite{myers2011}.

Myers \cite{myers2011} define os seguintes princípios para o teste de Software:

\begin{itemize}
  \item Testar é o processo de executar o programa com a intenção de encontrar erros;
  \item O teste é mais bem sucedido quando não realizado pelo próprio desenvolvedor;
  \item Um bom caso de teste é aquele que possui uma maior probabilidade de encontrar erros não conhecidos;
  \item Um caso de teste bem sucedido é aquele que detecta um erro desconhecido;
  \item Um teste bem sucedido inclui a definição cuidadosa dos dados de entrada e saída;
  \item Um teste bem sucedido inclui estudar cuidadosamente os resultados obtidos.
\end{itemize}

Teste está presente em todos os estágios do ciclo de desenvolvimento de
Software, mas é feito de uma maneira diferente em cada nível. Lu Luo
\cite{luo2001} define os seguintes níveis de teste:

\begin{itemize}
  \item \textbf{Teste Unitário:} Teste feito no mais baixo nível. Testa as estruturas unitárias do software, que são as menores partes testáveis;
  \item \textbf{Teste de Integração:} É realizado quando duas ou mais unidades são combinadas em uma estrutura maior;
  \item \textbf{Teste de Sistema:} Tende a afirmar a qualidade de todo o sistema. Esse teste é muitas vezes baseado em requisitos funcionais do sistema. Atributos de qualidade, tipicamente não funcionais, também são verificados;
  \item \textbf{Teste de Aceitação:} É um teste formal destinado a validar a aceitação do sistema diante da avaliação do usuário, cliente ou entidade autorizada.
\end{itemize}

Existem duas categorias de técnicas para teste de software. Diferentes
técnicas revelam diferentes aspectos de qualidade do software. As categorias
são funcional e estrutural. Os testes funcionais recebem dados de entrada, e
sua saída é avaliada para análise da conformidade com o esperado. O teste
estrutural também recebe dados de entrada e tem sua saída avaliada, porém, a
análise vai além, verificando a estrutura interna do sistema ou componente
\cite{luo2001}.

O teste funcional, ou caixa-preta, visualiza o sistema como uma caixa-preta. O
objetivo não é verificar o comportamento interno ou estrutural, mas sim em
quais circunstâncias o programa não se comporta como esperado \cite{myers2011}.

O teste estrutural, ou caixa-branca, permite examinar a estrutura interna do
programa. Essa estratégia deriva testes através da lógica do programa. O
objetivo é testar todos os caminhos possíveis de execução \cite{myers2011}.

  \subsection{Testes para Dispositivos Android}

O Android SDK (\textit{Software Development Kit}) \cite{androidSdk}, fornece um
conjunto de ferramentas para o desenvolvimento de aplicativos para a plataforma
Android. Entre elas, um \textit{framework} integrado de testes. Esse
\textit{framework} disponibiliza uma biblioteca de suporte para testes, ATSL
(\textit{Android Testing Support Library}) \cite{atsl}, que facilita o
desenvolvimento de testes para variados aspectos dos aplicativos.

    \subsubsection{O \textit{Framework} de Testes}

O \textit{framework} de testes possui como base dois módulos. Um módulo para
testes locais na JVM (\textit{Java virtual machine}) \cite{java}, e outro para
testes que precisam ser executados em um dispositivo Android, seja ele um
emulador ou um dispositivo físico.
O primeiro módulo do \textit{framework} de testes possui como objetivo o
suporte e execução de testes unitários, utilizando como base o JUnit
\cite{junit2015}. O segundo módulo possui como objetivo o suporte à testes de
integração. O suporte para testes de integração é alcançado através da
utilização de ferramentas externas, descritas no tópico
\ref{subsec:testedesoftwaresuptecnologico} de ferramentas de suporte.

\begin{figure}[H]
  \centering
  \includegraphics[width=10cm]{figuras/android_test.png}
  \caption{Estrutura do framework de testes \cite{androidTesting2015}}
  \label{figura:testes}
\end{figure}

A Figura \ref{figura:testes} resume a estrutura do \textit{framework} de
testes. É possível ver que a estrutura de classes de testes do Android é
baseada no JUnit. Portanto, os testes são organizados em classes, pacotes e
projeto. O \textit{framework} também disponibiliza um conjunto de métodos,
\textit{InstrumentationTestRunner}, que permite a execução de componentes do
Android sem as restrições dos ciclos de vida \cite{androidTesting2015}.

%Focar na plataforma JADE
%Discutir sobre o add-on framework

  \subsection{Qualidade de Software}
  \label{subsec:qualidadedesoftware}

No contexto de software, a qualidade é vista como um aglomerado de
características que são alcançadas para que o resultado obtido no
desenvolvimento atenda às necessidades do usuário, sendo elas explícitas ou não
\cite{rocha2001}. As características de qualidade de software podem ser
divididas em funcionais, e não funcionais. Características funcionais,
normalmente, podem ser medidas através de métricas que possuem critérios
objetivos, tornando possível a avaliação da qualidade \cite{meirelles2013}.

    \subsubsection{Métricas de Qualidade de Código Fonte}

Métricas de software é um termo que envolve várias atividades para a definição
de escalas e métodos que possuem como objetivo a identificação de parâmetros
que afetam o desenvolvimento do software \cite{metrics}.

As métricas de software podem ser agrupadas em objetivas e subjetivas. As
métricas objetivas possuem regras bem definidas, e possibilitam comparações
posteriores. Enquanto que as métricas subjetivas podem alterar de acordo com o
ator responsável pela sua coleta \cite{meirelles2013}.

Métricas de código fonte objetivas possuem características que permitem o
mapeamento dos seus resultados em intervalos para serem interpretados em uma
fase de análise \cite{meirelles2013}. Alguns exemplos são:

\begin{itemize}
  \item Número de atributos;
  \item Média de complexidade ciclomática por método;
  \item Respostas para uma classe;
  \item Número de filhos;
  \item Fator de acoplamento, e
  \item Complexidade estrutural.
\end{itemize}

\section{Considerações Finais do Capítulo}

As várias particularidades ligadas a \textit{Mobile Social Games} evidenciam
que este ambiente virtual possui diversas interações presentes no cotidiano das
pessoas. Tal fato torna esse formato de jogo atrativo. A quantidade de
jogadores com interesse nesse estilo é crescente, fazendo com que o mercado e
\textit{Mobile Social Games} esteja em crescimento constante.

A aplicação do paradigma MultiAgentes torna-se oportuna nesse contexto, dado que
expõe possíveis melhorias de Engenharia de Software. Dentre essas melhorias,
destacam-se: autonomia, adaptabilidade e flexibilidade. Essas melhorias podem
potencializar os ganhos em produtividade e redução de custos de
desenvolvimento, visto que as abstrações são mais próximas às necessidades de
Mobile Social Games.

Por fim, esse capítulo listou boas práticas de Engenharia de Software, com foco
em testes de software e qualidade de software. Essas boas práticas tornarão
possível melhorias no contexto de desenvolvimento e Engenharia de Software.
